% ----------------------- Proposed Solution ----------------------------------

\section{Proposed Solution}
% Explain your proposed work in details. Clearly state your specific contributions and reusable components deployed in the project with sources as applicable.
\noindent
The proposed work centers around "Analyzing Social Media Posts for Mental Health Disorder Detection," leveraging advanced data analytics and machine learning techniques to provide insights into the sentiments expressed in social media discussions related to mental health issues. This research is significant due to the increasing prevalence of mental health disorders globally and the role social media plays in shaping public perception and discourse around these issues. The core objective of this project is to develop a systematic approach to classify sentiments in tweets, thereby enabling better understanding and awareness of mental health conditions through social media analysis. Below, I outline the specific contributions and reusable components deployed in this project.

\subsection{Special Contributions}
\begin{itemize}
    \item \textbf{Dataset Acquisition and Preparation} :
    \noindent
    The initial step involved sourcing a high-quality dataset from Kaggle, specifically the Twitter sentiment dataset. This dataset comprises user-generated tweets containing sentiments related to various mental health issues, serving as the primary data source for analysis. The preparation phase included extensive data cleaning and preprocessing, wherein missing values were handled, duplicate entries removed, and text normalization performed. This crucial step ensured the dataset's integrity and suitability for subsequent analysis, allowing for a more accurate representation of sentiments.
    \item \textbf{Text Vectorization} :
    \noindent
    To enable machine learning models to interpret textual data, TF-IDF was implemented. This approach involved converting posts into a numerical format by creating a matrix representation of word frequencies across the dataset. The Scikit-learn library was instrumental in this process, offering functions for text vectorization and feature extraction. The reusable components for text preprocessing and vectorization were packaged into functions, allowing for easy application to future datasets or similar projects.
    \item \textbf{Implementation of Machine Learning Algorithms} :
    \noindent
    The project employed several machine learning algorithms, focusing primarily on Logistic Regression and XGboost for smental health classification. The Logistic Regression algorithm was chosen for its simplicity and effectiveness in classifying data points based on proximity in the feature space. In addition to Logistic Regression, XGboost was implemented due to its robustness in handling high-dimensional data and multi class classification tasks. 
    \item \textbf{Model Evaluation} :
    \noindent
    Comprehensive evaluation metrics were employed to assess the performance of the machine learning models. Metrics such as accuracy, precision, recall, and F1-score were calculated to provide a holistic view of model effectiveness in classifying sentiments related to mental health. This evaluation process not only demonstrated the models' capabilities but also highlighted areas for improvement, providing a foundation for future iterations of the project. The evaluation framework, including metrics calculations and visualization, was designed as reusable components to streamline future model assessments.
    \item \textbf{Insights and Recommendations} :
    \noindent
    A critical aspect of this project is the generation of actionable insights based on the analysis of social media sentiments. The findings from the classification can inform mental health professionals, researchers, and policymakers about public sentiment trends, potential stigma associated with mental health issues, and the effectiveness of awareness campaigns. Recommendations for mental health awareness strategies can be derived from understanding how sentiments vary across different demographics and regions. This interpretative analysis, combined with quantitative results, contributes valuable knowledge to the ongoing conversation about mental health in society.
    \item \textbf{Documentation and Reproducibility} :
    \noindent
    To enhance the usability and impact of the project, thorough documentation was maintained throughout the research process. This documentation includes detailed explanations of the methodologies employed, code snippets, and instructions for reproducing the results. The aim is to ensure that the components developed in this project can be easily utilized by other researchers and practitioners in the field. By documenting the code and methodologies, I am contributing to the open-source community, allowing for collaborative improvements and innovations in mental health sentiment analysis.
\end{itemize}

\subsection{Reusable Components}
\begin{itemize}
    \item \textbf{Data Collection Functions} :
    \noindent
    Modular functions designed for data collection, which can be reused across different platforms.
    \item \textbf{Data preprocessing Module} :
    \noindent
    A component that does data cleaning to remove the duplicates and empty rows and add a seperate column for cleaned texts. This formatted dataset is then used further with TF-IDF to create a numerical matrix to be fed to the machine leanring algorithms.
    \item \textbf{Machine and Deep Learning Model Functions} :
    \noindent
    Functions for implementing Logistic Regression, Naive Bayes, Support Vector Machine, Random Forest, XGboost, Long Short Term Memory algorithm allowing for easy retraining on varying datasets. These also features various evaluation metrics, making it easy to assess different models' performances.
    \item \textbf{Deployment Function} :
    \noindent
    A seperate function that has the main python file for creating web based application on Streamlit Cloud. This also includes the requirements and package dependencies for deploying the application.
\end{itemize}

% ----------------------- Proposed Solution Ends -----------------------------
