% -------------- Problem Definition and Preleminaries ------------------------
\pagebreak
\section{Problem Definition and Preliminaries}

\begin{comment}
    State the problem you have solved. Define the boundaries of your project. What was included (scope) and what was not (exclusions)?
\vspace{.1in}

\noindent
Briefly explain the methods you used to conduct your research or complete the project. Were there specific tools, techniques or libraries employed?
\end{comment}

\subsection{Context and Background}
\noindent
Mental health disorders have become a significant public health concern worldwide. The World Health Organization (WHO) estimates that approximately 1 in 8 people globally experience mental health disorders, which encompass conditions such as depression, anxiety, bipolar disorder, and post-traumatic stress disorder (PTSD). The rise of social media platforms has changed how individuals express their mental health struggles, share experiences, and seek support. Posts on platforms like Reddit and Twitter provide a wealth of data reflecting real-time sentiments, issues, and conversations surrounding mental health. However, this vast amount of unstructured textual data presents challenges in effectively identifying and categorizing specific mental health disorders.

\subsection{Objective}
\noindent
The primary objective of this project is to develop a robust system that can automatically analyze social media posts, specifically from Reddit and Twitter, to detect various mental health disorders. By leveraging Natural Language Processing (NLP) techniques and machine learning algorithms, this project aims to:
\begin{itemize}
    \item \textbf{Classify Posts} :
    \noindent
    Accurately classify social media posts based on the type of mental health disorder mentioned, including but not limited to depression, anxiety, bipolar disorder, and PTSD.

    \item \textbf{Data Driven Insights} :
    \noindent
    Provide valuable insights into the prevalence and expression of mental health issues on social media, helping researchers, mental health professionals, and policymakers understand trends and patterns.
\end{itemize}

\subsection{Challenges}
\begin{itemize}
    \item \textbf{Data Variability} :
    \noindent
    Social media posts can vary significantly in structure, style, and length. Users may employ slang, abbreviations, and informal language, making it difficult for algorithms to accurately interpret and classify posts.

    \item \textbf{Imbalanced Data} :
    \noindent
    Certain mental health issues may be underrepresented in social media discussions, leading to an imbalanced dataset. This imbalance can adversely affect model training and performance, making it harder to detect less frequent disorders.

    \item \textbf{Cultural and Contextual Nuances} :
    \noindent
    Mental health perceptions and discussions can vary across different cultures and contexts. The model needs to account for these nuances to avoid misclassification and provide accurate insights.

    \item \textbf{Privacy and Ethical Considerations} :
    \noindent
    Analyzing social media data raises ethical concerns regarding user privacy. It is crucial to handle sensitive information responsibly and comply with data protection regulations.
    
\end{itemize}

\subsection{Scope}
\noindent
The scope of the project "Analyzing Social Media Posts for Mental Health Disorder Detection" delineates the specific aspects that will be covered, the methodologies employed, and the boundaries within which the research and analysis will occur. The project aims to harness the potential of machine learning and natural language processing (NLP) techniques to analyze social media sentiment and its correlation with mental health disorders, focusing specifically on a Reddit Dataset sourced using Python Reddit API Wrapper. This dataset contains user-generated content that reflects various emotional states, making it a valuable resource for this analysis. The application allows user to input text, image, video, Reddit/Twitter username (using their API) and the model will classify the input into one of the mental health disorders (Normal, Anxiety, Depression, Bipolar, PTSD).

\begin{itemize}
    \item \textbf{Dataset Selection and Characteristics} \\
    \noindent
    The primary data source for this project is the top textual posts from Reddit. This dataset includes posts that are labeled with mental health issues (Normal, Anxiety, Depression, Bipolar, PTSD). The selection of this dataset is pivotal, as it encapsulates a wide range of mental health-related discussions expressed by individuals on social media. Key characteristics of the dataset include:
    \begin{itemize}
        \item \textbf{User Anonymity} :
        \noindent
        To respect user privacy and adhere to ethical standards, the dataset does not contain personally identifiable information (PII) about the Reddit users. This ensures compliance with data protection regulations while allowing for robust analysis.
    \end{itemize}

    \item \textbf{Methodologies} :
    \noindent
    The project will employ various methodologies to achieve its objectives, including:
    \begin{itemize}
        \item \textbf{Data Processing} :
        \noindent
        Cleaning and preparing the dataset to ensure that it is suitable for analysis. This includes tasks such as removing noise (e.g., URLs, hashtags), tokenization, and normalization of text.
        \item \textbf{Feature Extraction} :
        \noindent
        Utilizing techniques like Term Frequency-Inverse Document Frequency (TF-IDF) to convert textual data into numerical representations suitable for machine learning algorithms.
        \item \textbf{Machine Learning Techniques} :
        \noindent
        Implementing various machine learning algorithms, including Logistic Regression, Naive Bayes, SVM, LSTM, Transformer and XGboost to classify posts and evaluate their performance based on accuracy, precision, recall, and F1 score.
        \item \textbf{Data Visualizations} :
        \noindent
        Employing visualization tools to present findings clearly, including heat maps for confusion matrices and ROC AUC Curve.
    \end{itemize}
\end{itemize}

\subsection{Exclusions}
\noindent
In delineating the boundaries of the project "Analyzing Social Media Posts for Mental Health Disorder Detection," it is crucial to specify what is excluded from the scope of this research to maintain a clear focus on the primary objectives. Therefore, any analysis involving the dynamic aspects of social media engagement, such as real-time sentiment shifts in response to current events or trending topics, will be outside the project's purview. Furthermore, the study will not address the technicalities of Reddit’s platform-specific features, such as hashtags, user mentions, or reposts, subreddits in detail, as these elements are not central to the primary research focus on mental issue classification of individual users. While the project aims to analyze posts surrounding mental health, it will also include mapping the mental issue with mental wellbeing. Additionally, the research will not explore the ethical implications of data ownership or the responsibilities of social media platforms regarding user-generated content, as the focus will be primarily on data analysis techniques and outcomes rather than the broader ethical landscape. The web application also excludes the comments, subreddits and other metadata of the Reddit/Twitter posts for the sake of simplicity and focus on the main objective of mental health disorder detection.

\subsection{Assumptions}
\noindent
In the context of the project "Analyzing Social Media Posts for Mental Health Disorder Detection," several key assumptions underpin the research framework and methodologies employed. Firstly, it is assumed that the Reddit dataset obtained using PRAW is representative of broader social media discourse regarding mental health issues, capturing a diverse range of sentiments expressed by users on the platform. This assumption is critical as it establishes the foundation for analyzing sentiment trends and their potential correlations with various mental health disorders. Secondly, it is presumed that the posts recorded in the dataset accurately reflect the users' true emotions and perspectives at the time of posting, thereby providing valid data for analysis. Furthermore, it is assumed that the textual data within the dataset can be effectively processed and interpreted through natural language processing (NLP) techniques, allowing for the accurate classification of sentiments and identification of patterns. Another assumption is that the selected machine learning algorithms, including Logistic Regression, SVM, Naive Bayes, LSTM, Transformer and XGboost, will perform optimally with the provided data, leading to reliable and interpretable results regarding sentiment analysis and mental health correlations. Additionally, it is assumed that the sentiments expressed in social media posts can serve as a valid proxy for understanding public perceptions of mental health, enabling insights into societal attitudes and the potential stigmatization associated with these disorders. The project also assumes that the cleaning and preprocessing steps applied to the data will sufficiently prepare the dataset for analysis, minimizing noise and irrelevant information that could skew the results. Lastly, it is presumed that the ethical considerations surrounding the use of publicly available social media data have been adequately addressed, ensuring that the research adheres to relevant ethical standards and does not compromise user privacy or data integrity. These assumptions serve as the bedrock for the project's analytical framework, guiding the research processes and interpretations that follow.

% ----------- Problem Definition and Preliminaries ends ----------------------
