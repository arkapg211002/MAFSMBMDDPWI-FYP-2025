% ------------------------ Introduction -------------------------------------

\section{Introduction}

\begin{comment}
    Briefly introduce the project's overall topic and purpose.
    \vspace{.1in}
    
    \noindent
    Provide specifications of Technical domain (Hardware, Operating System, Software) and Business domain.
    \vspace{.1in}
    
    \noindent
    Provide \textbf{Glossary} / Keywords in a tabular format.
\end{comment}

% Real Input

\subsection{Project Overview}
\vspace{.1in}
\noindent
Mental health disorders—including depression, anxiety, bipolar disorder, and PTSD—affect millions of people worldwide, creating a growing public health concern. In today’s digital era, social media platforms have become a ubiquitous space where individuals freely express their emotions, share personal struggles, and document life’s challenges, often without realizing that these expressions can serve as early indicators of mental health issues. This project leverages advanced machine learning and deep learning techniques to analyze social media posts by examining intricate language patterns, contextual cues, and sentiment variations to detect signs of mental distress. By employing a multimodal approach, the framework not only processes text data through methods like TF-IDF, Bag-of-Words, and deep neural networks but also integrates additional data sources such as images, videos, PDFs, and user responses to visual stimuli. This integration also allows the system to extract textual information from PDFs using OCR, analyze emotions via specialized image processing modules, and incorporate a comprehensive wellbeing survey. The survey responses are then mapped to specific parameters of Ryff’s Wellbeing Scale using an association matrix, which provides personalized insights and actionable recommendations using Gemini and RAG for early intervention. Ultimately, this holistic framework is designed to facilitate timely support and guide individuals towards appropriate mental health services, contributing to a more proactive approach in managing mental health on a global scale.

\subsection{Project Purpose}
\vspace{.1in}
\noindent
This project aims to develop a web application using machine learning and deep learning models (Logistic Regression, Naive Bayes, SVM, XGBoost, LSTM, Transformer) to detect mental health disorders from text, images, videos, PDFs, user responses to images, and social media profiles. By addressing technical challenges in processing large datasets and optimizing algorithms, the project seeks to enable early detection of mental health issues, contributing to public health improvement through technology.

\subsection{Technical Domain Specifications}
\vspace{.1in}
This project falls within the intersection of natural language processing (NLP) and machine learning (ML), leveraging techniques such as text vectorization, and classification algorithms. Here are the key technical domain specifications:

\newpage

\begin{itemize}
    
    \item \textbf{Hardware} : 
    \noindent
    The project does not require specialized hardware beyond a standard machine with adequate processing power. However, for larger datasets or complex model training, a machine equipped with a GPU (Graphics Processing Unit) could significantly reduce processing time. The project can be run on any system with at least 8GB of RAM and a multi-core processor.

    \item \textbf{Operating System} : 
    \noindent
    The project is cross-platform and can be developed and executed on any modern operating system, including macOS, Windows 10/11, Linux distributions (Ubuntu, Linux Mint). 

    \item  \textbf{Software} :
    \noindent
        \begin{itemize}
            \item \textbf{Programming Languages} : 
            \noindent
            Python 3.x will be the primary programming language, given its extensive libraries for machine learning, data analysis, and NLP.

            \item \textbf{Libraries / Frameworks} :
            \noindent
            This project leverages a comprehensive suite of libraries and frameworks to enable its functionality, including data processing (Pandas, NumPy), machine learning (Scikit-learn, XGBoost, TensorFlow, Transformers), image and text analysis (OpenCV, Tesseract, Pytesseract, DeepFace), audio processing (Librosa, PyDub, SpeechRecognition), social media integration (PRAW, Tweepy), and visualization (Plotly, Matplotlib). Additional tools like Streamlit, NLTK, and Google Generative AI ensure seamless application development, multilingual support, and advanced model implementation.    

            \item \textbf{Development Environment} :
            \noindent
            Google Colab is used for cloud-based execution when working with larger datasets or GPU-based model training.
                

                
        \end{itemize}
    
\end{itemize}

\subsection{Business Domain Specifications}
\noindent
From a business perspective, this project offers significant value in mental health monitoring, public health awareness, and social media governance. As mental health issues rise globally, organizations seek innovative solutions to address this crisis. By using machine learning to detect mental health disorders from social media data, this project can transform mental health management at both individual and societal levels, impacting various industries effectively.

\begin{itemize}
    
    \item \textbf{Mental Health Services} :
    \noindent
    Mental health providers, including hospitals and therapy centers, can leverage machine learning to detect early signs of mental disorders from social media data. This proactive approach complements traditional self-reporting and clinical assessments, enabling earlier intervention and support for patients.

    \item \textbf{Social Media Platforms} :
    \noindent
    Social media platforms like Twitter and Reddit are key spaces for expressing thoughts and emotions, including mental health struggles. This project’s machine learning models can help these platforms safeguard user well-being by identifying concerns early, while maintaining ethical standards.

    \item \textbf{Public Health Organizations} :
    \noindent
    Public health organizations can use real-time social media data to monitor mental well-being, identify trends, and design data-driven interventions. By analyzing language patterns, they can create targeted awareness campaigns that better engage individuals facing mental health challenges.
    
\end{itemize}


\subsection{Glossary / Keywords}
\noindent

\begin{center}

\begin{tabular}{|p{4cm}|p{10cm}|}
  \hline
  \multicolumn{1}{|c|}{\textbf{Term}} & \multicolumn{1}{c|}{\textbf{Definition}} \\

  \hline 
  Natural Language Processing (NLP) & A branch of artificial intelligence focused on the interaction between computers and humans through natural language, including tasks like text analysis. \\

  \hline 
  Retrieval-Augmented Generation (RAG) & A hybrid NLP framework that combines information retrieval and text generation by fetching relevant context from a knowledge base before generating responses, improving factual accuracy and relevance. \\


  \hline 
  Vectorization & The process of converting textual data into numerical form (such as a vector) so that it can be used as input for machine learning models. \\

  \hline 
  Classifier & A machine learning model or algorithm that categorizes or labels data points into predefined classes. \\

  \hline
  Mental Health Disorder & A wide range of conditions that affect mood, thinking, and behavior, including depression, anxiety, schizophrenia, etc. \\

  \hline
  Data Preprocessing & The process of preparing raw data for analysis by cleaning, normalizing, and transforming it into a usable format for machine learning models. \\

  \hline 
  Cross-validation & A model validation technique used to assess how well a model performs by dividing data into training and testing sets multiple times for better accuracy. \\

  \hline
  Precision & In the context of classification, precision refers to the accuracy of positive predictions, calculated as the ratio of true positives to the sum of true and false positives. \\

  \hline
  Recall & In classification, recall measures the ability of a model to identify all relevant instances within a dataset, calculated as the ratio of true positives to the sum of true positives and false negatives. \\
  
  \hline
  PRAW & PRAW (Python Reddit API Wrapper) is a Python library that provides a simple interface to interact with Reddit's API for accessing Reddit data, such as posts, comments, and user information. \\
  \hline
\end{tabular}

\begin{tabular}{|p{4cm}|p{10cm}|}
    \hline
    \multicolumn{1}{|c|}{\textbf{Term}} & \multicolumn{1}{c|}{\textbf{Definition}} \\
  

  \hline
  TesseractOCR & TesseractOCR is an open-source Optical Character Recognition (OCR) engine that extracts text from images with high accuracy; it is widely used for various applications like scanning documents and digitalizing printed text. \\

  \hline
  Depression & There is a difference between depression and mood swings or short-lived emotional reactions to daily experiments; A mental state causing painful symptoms adversely disrupts normal activities (e.g., sleeping). \\

  \hline
  Anxiety & Several behavioral disturbances are associated with anxiety disorders, including excessive fear and worry. Severe symptoms cause significant impairment in functioning cause considerable distress. Anxiety disorders come in many forms, such as social anxiety, generalized anxiety, panic, etc. \\

  \hline
  Bipolar Disorder & An alternating pattern of depression and manic symptoms is associated with bipolar disorder. An individual experiencing a depressive episode may feel sad, irritable, empty, or lose interest in daily activities. Emotions of euphoria or irritability, excessive energy, and increased talkativeness can all be signs of manic depression. Increased self-esteem, decreased sleep need, disorientation, and reckless behavior may also be signs of manic depression. \\

  \hline
  Post-Traumatic Stress Disorder (PTSD) & In PTSD, persistent mental and emotional stress can occur after an injury or severe psychological shock, characterized by sleep disturbances, constant vivid memories, and dulled response to others and the outside world. People who re-experience symptoms may have difficulties with their everyday routines and experience significant impairment in their performance. \\

  \hline
  DeepFace & DeepFace is a Python library for deep learning-based facial recognition and attribute analysis. It supports several pretrained models and simplifies face recognition tasks, making it suitable for various applications in image analysis. \\

  \hline
  Transformers Module & The Transformers module in Python, developed by Hugging Face, is a library for natural language processing (NLP) tasks like text classification, translation, and summarization, using state-of-the-art models like BERT and GPT. \\

  \hline
  Gemini 2.0 Flash & Gemini 2.0 Flash is a cutting-edge AI model developed by Google, capable of performing advanced generative and analytical tasks across text, image, and other modalities. \\

  \hline
  FFmpeg & FFmpeg is a multimedia framework used for encoding, decoding, transcoding, streaming, and manipulating audio and video files, supporting a wide range of formats and codecs. \\

  \hline
  Hyperparameter Tuning & Hyperparameter tuning involves selecting the best parameters for a machine learning model to optimize its performance on a given task, using methods like grid search or random search. \\
  \hline
\end{tabular}
  
\end{center}


% ------------------------------ Introduction Ends ---------------------------
